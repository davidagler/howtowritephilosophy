David W. Agler\\
Class or Assignment Name\\
Date Submitted

\section{Paper Title: A Great Simple
Essay}\label{paper-title-a-great-simple-essay}

\subsection{Section 1. Introduction}\label{section-1.-introduction}

In this paper, I will argue that X is the case. In section 2, I will
explain problem / concept / theory Y. After explaining this problem, in
section 3, I will provide an argument that X is the case. Finally, in
section 4, I will summarize my results and point to limitations and
areas where further research is needed.

\subsection{Section 2. Clarification of a
Problem}\label{section-2.-clarification-of-a-problem}

In this section, I will clarify a problem / concept / theory Y.

\begin{quote}
``Nunc sed pede. Praesent vitae lectus. Praesent neque justo, vehicula
eget, interdum id, facilisis et, nibh.'' (Agler 2024, p.~34)
\end{quote}

Next, I will explain the above passage. This passage makes two key
claims. First, when the author says X, what is meant is {[}\ldots{]}.
Second, it says \ldots{}

To illustrate the first claim, it is helpful to think of the following
example {[}\ldots{]}

\subsection{Section 3. Argument}\label{section-3.-argument}

Now that problem Y has been clarified, I will argue that Z is the case.
To support this claim, consider the following argument:

\begin{itemize}
\tightlist
\item
  P1: A proposition
\item
  P2: A proposition
\item
  C: Z is the case (conclusion)
\end{itemize}

P1 says A. This premise is true because of C. P2 says B. This premise is
true because of D.

In this section, I have shown that Z is the case.

\subsection{Section 4. Conclusion}\label{section-4.-conclusion}

In this paper, I have shown that there is a powerful argument in support
of Z.

\subsection{References}\label{references}

\begin{enumerate}
\def\labelenumi{\arabic{enumi}.}
\tightlist
\item
  My Great Reference Citation
\item
  Another Great Reference Citation
\end{enumerate}
