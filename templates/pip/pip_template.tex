David W. Agler\\
Class or Assignment Name\\
Date Submitted

\section{Paper Title: PIP}\label{paper-title-pip}

\subsection{Section 1. Introduction}\label{section-1.-introduction}

In this paper, I will explain X. In Section 2, I will first quote a
passage from Resource Y. After quoting this passage, I will explain the
passage and provide an illustration. In section 3, I will take a stance
on this claim.

\subsection{Section 2. Passage, Explanation, and
Example}\label{section-2.-passage-explanation-and-example}

In this section, I will interpret the following passage of text:

\begin{quote}
``Nunc sed pede. Praesent vitae lectus. Praesent neque justo, vehicula
eget, interdum id, facilisis et, nibh. Phasellus at purus et libero
lacinia dictum. Fusce aliquet. Nulla eu ante placerat leo semper dictum.
Mauris metus. Curabitur lobortis. Curabitur sollicitudin hendrerit nunc.
Donec ultrices lacus id ipsum.'' (Agler 2024, p.~34)
\end{quote}

Next, I will explain the above passage. This passage makes two key
claims. First, when the author says X, what is meant is {[}\ldots{]}.
Second, it says \ldots{}

To illustrate the first claim, it is helpful to think of the following
example . . . .

\subsection{Section 3. Stance}\label{section-3.-stance}

The view that X is Y is mistaken for reason Z. In support of reason Z, I
present the following argument {[}Insert Argument Here{]}. P1 of this
argument says ABC. P2 of this argument says DEF.

\subsection{References}\label{references}

\begin{enumerate}
\def\labelenumi{\arabic{enumi}.}
\tightlist
\item
  My Great Reference Citation
\item
  Another Great Reference Citation
\end{enumerate}
